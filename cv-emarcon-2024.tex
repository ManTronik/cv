\documentclass[11pt,a4paper]{moderncv}
\moderncvtheme[blue]{classic}                
\usepackage[utf8]{inputenc}
\usepackage[inline]{enumitem}
\usepackage[french]{babel}

\usepackage{perpage}
\MakePerPage{footnote}

\usepackage[top=1.0cm, bottom=2.0cm, left=1.6cm, right=1.6cm]{geometry}
% Largeur de la colonne de gauche pour les dates
\setlength{\hintscolumnwidth}{2.7cm}

\firstname{Emmanuel}
\familyname{Marcon}
\title{Architecte Sénior}              
\address{7 Ter Rue Paul Leboucher}{95240 Cormeilles-en-Parisis}    
\email{manu.marcon@orange.fr}                      
\email{emmanuel.marcon@orange.com}
\mobile{06 30 74 41 56} 
\extrainfo{53 ans -- marié 3 enfants }
\photo[64pt][0.4pt]{photo.jpg}

\begin{document}

\maketitle
% Marge négative entre le titre et la partie expérience, pour gagner de la place
\vspace*{-2.5\baselineskip}


\section{Expériences}
\cventry{2020\\à aujourd'hui}{Directeur Technique du programme IPv6 }{}{Orange France }{}{
\begin{itemize}%
\item \textbf{accélérer le déploiement du dual-stack au sein des infrastructures PFC }
  \begin{itemize}
   \item pilotage de l'activité , vulgarisation et documentation sur l'IPv6 
   \item gestion de la pénurie IPv4 , formation au processus d'attribution d'IPv4  de deux personnes à Rabbat (Maroc) 
   \item création d'un processus outillé ( perl + python + pandas) de récupération d'adresse IPv4  ( projet CUBE ) 
   \item certifié Yellow Belt : sujet amélioration de la qualité de données dans l'IPAM Grapa   \newline
\end{itemize}
\end{itemize}
}

\cventry{2016\\à 2020}{Architecte Solution Cloud}{}{entité Orange France}{}{
\begin{itemize}%
 \item \textbf{construction du cloud Privé BRMC } 
  \begin{itemize}%
   \item sélection du partenaire via RFQ / RFP, dépouillement et choix du partenaire
   \item build de la solution, membre d'une équipe agile,  développement des services load-balancer et accès Internet 
   \item approche systémique et modélisation MBSE ( Model Based System Engineering ) de l'offre IaaS BRMC \newline
 \end{itemize}
 \end{itemize}}

\cventry{2013\\à 2016}{Technical Design Authority}{}{entité Orange Cloud For Business - Paris Médéric  }{}{
\begin{itemize}%
\item \textbf{responsable du design} d'un cloud publique Flexible Computing Advanced sur Hong Kong et Singapour ( opex/capex 15M€ euros ) 
  \begin{itemize}%
   \item responsable du Build de la solution : 3 mois de design avec 10 ingénieurs / animation de 30 ateliers sur Rennes 
   \item solution VMWAre avec redondance bi-sites hautement redondante 
   \item formation des équipes opérationnelles en Inde ( 30 personnes formées ) sur l'exploitation de la solution 
   \item certificationTOGAF (level 1) et certifié Overall Architect (Orange IMT) en 2015    \newline
\end{itemize}
\end{itemize}}

\cventry{2011\\à 2013}{Support avant-vente offre cloud}{}{entité Orange Cloud for Business - Paris Médéric }{}{
\begin{itemize}%
\item \textbf{support technique} sur les solutions cloud publique d'OCfB
  \begin{itemize}%
   \item support de la migration d'un client avec intervention sur site en Chine 
   \item proof-of-concept de la solution Cisco VMDC ( Virtual MultiService DataCenter) , Cisco Fabric IP + BMC Cloud Lifecycle Management
   \item anticipation sur Openstack    \newline
\end{itemize}
\end{itemize}}

\cventry{2007\\à 2011}{Ingénieur Solution Et Conception}{}{entité Orange Business Services } {}{
\begin{itemize}%
\item \textbf{référent technique déploiement } pour le déploiement de la solution de video-conferencing \newline Telepresence Cisco 
  \begin{itemize}%
   \item développement en Perl d'un outil de vérification des pré-requis network ( jitter, latency et graph rrdtool ) 
   \item responsable du design d'un backbone dédié IPv4 ( Prise IP - projet LVMH) puis industrialisation de la migration des sites 
   \item expert Cisco Cisco Certified Network Professional     \newline
\end{itemize}
\end{itemize}}


\cventry{2003\\à 2007}{Ingénieur Réseau }{}{entité Transpac à Ivry (94) } {}{
\begin{itemize}%
   \item responsable (VPN owner) du VPN du groupe Total (1200 sites worldwide) 
   \item référent sur les activés de Run + expertise sur les technologies MPLS et VPN  \newline 
\end{itemize}}


\cventry{2000\\à 2003}{Network Analyst chez Sita/Equant Neuilly(92)}{}{} {}{
\begin{itemize}%
   \item responsable (VPN owner) du VPN du groupe Atofina (500 sites worldwide) 
   \item gestion du cycle de vie avec le maintient des cohérences des configurations 
   \item création d'un script en PERL pour tests automatiques des liaisons de secours ISDN entrainant un gain significatif sur les Opex  \newline
\end{itemize}}


\cventry{1999\\à 2000}{Ingénieur réseau chez CS Experdata Paris} {}{ } {}{
\begin{itemize}%
   \item raccordement de l'université Basse Normandie à RENATER - projet Vikman 
   \item déploiement d'une salle de marché suite à la fusion du CIC et du crédit mutuel ( 2000 postes dont 350 pour les trader)  \newline
\end{itemize}}

\cventry{1998\\à 1999}{Ingénieur réseau chez GFI Informatique}{}{} {}{
  \begin{itemize}%
   \item administrateur du réseau token-ring du siège de Danone à Paris 
   \item gestion des incidents , audit du réseau puis amélioration de la QOS (Quality Of Service)   \newline
  \end{itemize}}


\cventry{1997\\à 1998}{Coordinateur de projets}{}{Cegetel,  groupe Vivendi - Paris-La-Défense} {}{
  \begin{itemize}%
   \item coordination des migrations des clients sur les nouvelles offres data (frame-relay , internet) et voix \newline
  \end{itemize}}

\cventry{1994\\à 1997}{Responsable technique}{}{Santerne SA , groupe Vivendi à Arras(62) } {}{
\begin{itemize}%
  \item déploiement d'infrastructure réseaux locaux ( ethernet, token-ring  ) 
   \item audit réseau Ethernet et Token Ring , analyse et actions correctives 
   \item déploiement des software de supervision (HP Openview)   \newline
\end{itemize}}


\section{Formation}
\cventry{}{} {}{} {} {
\begin{itemize}
 \item \textbf{ 1993 } - Licence Informatique Industrielle et Automatique (Hautes Etudes Industrielles Lille) 
 \item \textbf{ 1992 } - Service militaire - chasseur Alpin (sous-officier) dans le 27RCS Grenoble 
 \item \textbf{ 1991 } - DUT Génie Electrique et informatique industrielle (IUT de Béthune) 
 \item \textbf{ 1989 } - Baccalauréat série E (Mathématiques et Technologies - Lycée Carnot Arras)   \newline
 \end{itemize}}

\section{Autres centres d'intérêt}
\cventry{}{} {}{} {} {
\begin{itemize}
 \item  Anglais lu , écrit et parlé   
 \item  sportif avec la pratique régulière de vélo, piscine et course à pied  
 \item  montage électronique et programmation micro-controleur ( Arduino , ESP32.. ) 
 \item  shell script, perl, python 
 \item  pratique du Yoga et du dessin 
 \end{itemize}}
 
\vfill
\enlargethispage{\footskip}
\footnotetext{TOGAF : The Open Group Architecture Framework} 

\end{document}

